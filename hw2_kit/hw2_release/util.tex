%\usepackage[T1]{fontenc}
\usepackage[latin9]{inputenc}
\usepackage[letterpaper]{geometry}
\geometry{verbose,tmargin=1in,bmargin=1in,lmargin=1in,rmargin=1in}
\usepackage{babel}
\usepackage{amsmath}
\usepackage{amssymb}
\usepackage{capt-of}
\usepackage{graphicx}
\usepackage{color}
\usepackage{latexsym}
\usepackage{xspace}
\usepackage{pdflscape}
\usepackage[hyphens]{url}
\usepackage[colorlinks]{hyperref}
\usepackage{enumerate}
\usepackage{ifthen}
\usepackage{float}
\usepackage{array}
\usepackage{tikz}
\usepackage{multirow} 
\usetikzlibrary{shapes}
\usepackage{algorithm2e}
\usepackage{listings}
%%%% HW instructions / collaboration text

\newcommand{\HWPoliciesmix}{
\paragraph*{Instructions.} {\bf This homework contains two parts. Part A is written assignment and Part B is MATLAB programming assignment.}
}

\newcommand{\HWPolicies}{
\paragraph*{Instructions.} 
Please write up your responses to the following problems clearly and
concisely. We encourage you to write up your responses using \LaTeX{};
we have provided a \LaTeX{} template, available on Canvas, to
make this easier. {\bf Submit your answers in PDF form to Canvas. We will not accept paper copies of the homework.}

\paragraph*{Collaboration.} 
You are allowed and encouraged to work together. You may discuss the
homework to understand the problem and reach a solution in groups up
to size {\bf two students.} However, {\em each student must write
  down the solution independently, and without referring to written
  notes from the joint session. {\bf In addition, each student must
    write on the problem set the names of the people with whom you
    collaborated.}} You must understand the solution well enough in
order to reconstruct it by yourself. (This is for your own benefit:
you have to take the exams alone.)
}
\newcommand{\ProgrammingPolicies}[1]{
\paragraph*{Instructions.} 
This is a MATLAB programming assignment. This assignment consists of
multiple parts. Portions of each part will be graded automatically,
and you can submit your code to be automatically checked for
correctness to receive feedback ahead of time.

We are providing you with the codebase, templates, and dataset that you will
need for this assignment. Download the file {\tt hw#1\_kit.zip}
from Canvas {\bf before} beginning the assignment.  {\bf Please
  read through the documentation provided in ALL Matlab files before
  starting the assignment.} 

Note that you will submit a pdf with your answers to canvas, and will submit (and run) your matlab
code (and get feedback on it) as described on the wiki:

\begin{center}
\url{http://alliance.seas.upenn.edu/~cis520/wiki/index.php?n=Resources.HomeworkSubmission}
\end{center}

If you are not familiar with Matlab or how Matlab functions work, you can refer to Matlab online documentation for help:
\begin{center}
\url{http://www.mathworks.com/help/matlab/}
\end{center}

In addition, please use built-in Matlab functions rather than external
library functions. Without proper reference to an external library,
the auto-grader may fail even if your code runs perfectly on your local
machine. Also, please DO NOT include data files in your submission.

\paragraph*{Collaboration.} 
You are allowed and encouraged to work together. You may discuss the
homework to understand the problem and reach a solution in groups up
to size {\bf two students.} Please submit {\bf one copy} of your work. Be sure to include {\bf your and your collaborator's} pennkey and name in the {\bf group.txt} file. {\bf We will be using automatic checking software to detect blatant copying of other groups' assignments, so,
please, don't do it.}
}

%%%% CUSTOM COMMANDS FOR FORMATTING EXAMS/HOMEWORKS
\newcounter{section_points}[section]
\newcounter{header_points}[section]
\newcounter{total_points}

\newcommand{\hpoints}[1]{
  \setcounter{header_points}{#1}
  \textbf{[#1 points]}
}

\newcommand{\points}[1]{
  \addtocounter{section_points}{#1}
  \addtocounter{total_points}{#1}
  \textbf{[#1 
    \ifthenelse{\equal{#1}{1}}
    {point}{points}]}
}

\newcommand{\bpoints}[1]{
  \textbf{[#1 
    \ifthenelse{\equal{#1}{1}}
    {point}{points}]}
}


\newcommand{\point}{\textbf{[1 point]}}

\newboolean{ShowSolutions}
\newcommand{\Mistake}[2]{
  \ifthenelse{\boolean{ShowSolutions}}
  {\paragraph{\bf $\blacksquare$ COMMON MISTAKE #1:} {\sf #2} \bigskip}
  {}
}

\newcommand{\Solution}[2]{
  \ifthenelse{\boolean{ShowSolutions}}
    {
      \paragraph{\bf $\bigstar $ SOLUTION:} { \sf
        #1} \bigskip
    }
    { 
      #2
    } %} \vspace{1.5in}}
}
\newcommand{\out}[1]{}

\newboolean{ShowPointsInfo}

\newcommand{\PointStats}[0]{
  \ifthenelse{\boolean{ShowPointsInfo}}
  {
    \begin{center}

      \begin{tabular}{rl}
        \hline
        Stated Points: & \arabic{header_points} \\
        Section Points: & \arabic{section_points} \\
        Total Points So Far: & \arabic{total_points} \\
        \hline 
        \multicolumn{2}{c}{
          
          \ifthenelse{
            \equal{\value{section_points}}{\value{header_points}}
          }{CORRECT TOTAL}
          {{\bf INCORRECT TOTAL}}
          }
          \\
        \hline
      \end{tabular}
    \end{center}
  }{}
}

\newcounter{blankcount}
\newcommand{\myrepeat}[2] {
\setcounter{blankcount}{1}
\whiledo{\value{blankcount} < #1}{
#2
\addtocounter{blankcount}{1}
}
}

\newcommand{\blank}[1]{\underline{\myrepeat{#1}{\qquad}}}


%%%% CUSTOM MATH GOES HERE

\newcommand{\ind}[1]{\mathbf{1}\left(#1\right)}
\renewcommand{\Pr}{\mathbf{Pr}\xspace}
\newcommand{\Bern}{\textsf{Bernoulli}\xspace}
\newcommand{\sign}{\textsf{sign}}

\newcommand{\E}{\mathbf{E}}
\newcommand{\bx}{\mathbf{x}}
\newcommand{\bX}{\mathbf{X}}
\newcommand{\by}{\mathbf{y}}
\newcommand{\bY}{\mathbf{Y}}
\newcommand{\bz}{\mathbf{z}}
\newcommand{\bw}{\mathbf{w}}
\newcommand{\bl}{\mathbf{\ell}}
\newcommand{\vc}[1]{\mathbf{#1}}

\newcommand{\Hypo}{\mathcal{H}}
\newcommand{\XX}{\mathcal{X}}
\newcommand{\cD}{\mathcal{D}}

\newcommand{\argmax}{\operatornamewithlimits{argmax}}
\newcommand{\argmin}{\operatornamewithlimits{argmin}}

\newcolumntype{M}{>{$\vcenter\bgroup\hbox\bgroup}c<{\egroup\egroup$}}

\newcolumntype{x}[1]{>{\centering\arraybackslash}m{#1}}




